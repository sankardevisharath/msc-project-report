%%%%%%%%%%%%%%%%%%%%% RJT TeX Template %%%%%%%%%%%%%%%%%%%%%
\documentclass[twoside,11pt,a4paper]{article}

%%%% BHAM PREAMBLE - SET THIS FIRST! %%%%
\newcommand{\bhamstudentname}{Sarathkumar Padinjare Marath Sankaranarayanan}
\newcommand{\bhamthesistitle}{Machine Learning \& Deep Learning Approaches to Predict Credit Card Default}
\newcommand{\bhamfronttitle}{Machine Learning \& Deep Learning Approaches \\ to Predict Credit Card Default}
\newcommand{\bhamschool}{School of Computer Science}
\newcommand{\bhamcollege}{Engineering and Physical Sciences}
\newcommand{\bhamdegree}{MSc. Artificial Intelligence \& Computer Science}
\newcommand{\bhamid}{2359859}
\newcommand{\bhamsupervisor}{Dr.Kashif Rajpoot}
\newcommand{\bhamyear}{2022}
%%%%           %%%%


\usepackage[hyphens]{url}
\usepackage[breaklinks]{hyperref}
\usepackage{fancyhdr}
\usepackage[sort]{natbib} 
\usepackage{comment} % from http://www.latex-community.org/forum/viewtopic.php?f=5&t=4538
\usepackage{dirtree} % from http://blog.plenz.com/2011-07/represent-directory-structures-in-latex.html
\usepackage{longtable} % from http://stackoverflow.com/questions/2896833/how-to-stretch-a-table-over-multiple-pages
\usepackage{algorithm}   
\usepackage{algorithmic}   %both algorithm* from http://hasini-gunasinghe.blogspot.co.uk/2014/02/presenting-algorithmsprotocols-in-neat.html

\renewcommand{\algorithmiccomment}[1]{#1} %http://tex.stackexchange.com/q1uestions/61861/algorithmic-package-for-loop-and-comment-at-the-same-line
\usepackage[printonlyused]{acronym}
\usepackage[table]{xcolor}
\usepackage{subcaption}
\usepackage{placeins}
\hypersetup{
	colorlinks,
	linkcolor={green!50!black},
	citecolor={blue!50!black},
	urlcolor={blue!80!black}
}
\pagestyle{fancy}
\renewcommand{\sectionmark}[1]{\markboth{#1}{}}	%from tex.stackexchange.com/questions/111361

\lfoot{\bhamstudentname}
\cfoot{}
\rfoot{}
\fancyhfoffset[L]{0cm} %this fixes the right page number margin issue.
\newcommand{\HRule}{\rule{\linewidth}{0.5mm}}
\renewcommand{\headrulewidth}{0pt}
\newcommand{\tab}{\hspace*{1.25em}}
\newcommand{\minitab}{\hspace*{0.25em}}

%footnote stuff
\usepackage{perpage}
\MakePerPage{footnote} %the perpage package command
\renewcommand*{\thefootnote}{\fnsymbol{footnote}}

\lhead{}\chead{}\rhead{}
\setlength{\headheight}{28pt} %fixes the warnings about headheight being too small
\setlength{\headsep}{6pt}
\pdfoutput=1 % we are running PDFLaTeX
\usepackage[left=2.55cm,right=1.6cm,top=1.8cm,bottom=1.8cm]{geometry}
\usepackage{titling}	
\setlength{\droptitle}{-2.75cm}   % This is your set screw\\
\usepackage{titlesec}
\titleformat*{\section}{\normalsize	\bfseries}
\titleformat*{\subsection}{\small \bfseries}
\titleformat*{\subsubsection}{\footnotesize \bfseries}
%modifies the size of the gaps between the top of the title and bottom
% arguments {type}{left}{top}{bottom}
\titlespacing*{\section} {0pt}{3ex plus 1ex minus .2ex}{2ex plus .2ex}
\titlespacing*{\subsection} {0pt}{2.25ex plus 1ex minus .2ex}{0.75ex plus .2ex}
\titlespacing*{\subsubsection}{0pt}{2.ex plus 1ex minus .2ex}{0.5ex plus .2ex}

\setlength{\intextsep}{0pt} %from http://tex.stackexchange.com/questions/25828/how-to-remove-change-the-vertical-spacing-before-and-after-an-algorithm-environ

\usepackage[pdftex]{graphicx}
\graphicspath{ {figures/} }


\usepackage{enumitem}
\usepackage{pdfpages}
\usepackage{lastpage}
\usepackage{amsmath}
\usepackage{amsfonts}
\usepackage{amssymb}


\usepackage{epstopdf} %http://dirkraffel.com/2007/11/19/include-eps-files-in-latex/comment-page-2/

\usepackage{listings}
\lstset{
 basicstyle=\ttfamily,
  columns=fullflexible,
  keepspaces=true,
breaklines=true
} %from http://tex.stackexchange.com/questions/121601/automatically-wrap-the-text-in-verbatim

\newcommand{\todo}[1]{\textcolor{red}{TODO: #1}\PackageWarning{TODO:}{TODO found: #1!}} %from http://tex.stackexchange.com/questions/9796/how-to-add-todo-notes

\DeclareGraphicsExtensions{.jpg,.jpeg,.png,.ppm}
%%%%%%%%%%%%%%%%%%%%%  END of TEMPLATE %%%%%%%%%%%%%%%%%%%%%

\title{MSc. Project\\\bhamthesistitle}
\author{\textsf{\bhamstudentname {\textsf{}}}}

\date{}
\begin{document}

\pagenumbering{gobble} % fix at	 http://tex.stackexchange.com/questions/7355/how-to-suppress-page-number
% this came from http://en.wikibooks.org/wiki/LaTeX/Title_Creation and http://tex.stackexchange.com/questions/14778/error-with-hrule
\begin{titlepage}
\begin{center}
% this was from http://tex.stackexchange.com/questions/7219/how-to-vertically-center-two-images-next-to-each-other
\begin{minipage}{6in}
  \centering
  \raisebox{-0.5\height}{\includegraphics[width=1.25in]{crest}}
  \hspace*{.2in}
  \raisebox{-0.5\height}{\includegraphics[height=0.9375in]{uni}}
  \end{minipage}
  \\ [1.0cm]
\textsc{{\LARGE \bhamschool\\}College of \bhamcollege}\\[3.5cm]

\textsc{\Large MSc. Project}\\[0.5cm]

% Title
\HRule \\[0.4cm]
\begin{center}\Huge
\bhamfronttitle
\end{center}
\HRule \\[1.5cm]
% Team and Members

\begin{center}
Submitted in conformity with the requirements\\ for the degree of \bhamdegree\\
\bhamschool\\ University of Birmingham\\
\vspace{2cm}
\bhamstudentname \\
Student ID: \bhamid\\
Supervisor: \bhamsupervisor      
\end{center}
\vfill

% Bottom of the page
{\large September \bhamyear}

\end{center}
\end{titlepage}

\section*{\centering Abstract}	
Many studies related to Credit Card Default prediction have been conducted over the years; however, the datasets used in these studies are old and might not represent the current industry scenario. Due to the confidential nature of the data, there are not many Credit Card Default Prediction datasets are available in public domain and hence most of the studies ends up using the same old dataset. This study uses publicly available "American Express-Default Prediction" dataset which contains 5 Millon data records of 400,000 unique customers for the period 2018-19, to explore various Machine Learning, Ensemble Learning \& Deep Learning techniques for credit card default prediction problem. The dataset was trained on \acf{SVM}, \acf{RF}. \acf{GBDT}, \acf{XGBoost},\acf{LGBM}, \acf{ANN}, \acf{GRU}, and \acs{GRU}+\acs{ANN}+\acs{GBDT} models. \acs{LGBM} (81.01\%) \& \acs{ANN}(81.03\%) models provided highest F1-Scores while \acs{RF} (73.81\%) performed worst among all the models. Finally, a Lean \acs{LGBM} model was proposed for the industry  credit card default prediction use cases. The proposed model performs on par with \acs{LGBM} \& \acs{ANN} model and provides an F1-Score of 80.95\% while using less features and computational resources.\\



\textbf{Keywords} Credit Card Default Prediction, Ensemble Learning, Deep Learning

\vfill
\clearpage

\section*{\centering Acknowledgements}
To begin, I would like to thank my supervisor, Dr. Kashif Rajpoot, for his help, guidance and
advice throughout this project. Furthermore, I would like to thank Dr.Andrew NG whose Machine Learning course inspired me to pursue the MSc in AI \& CS.
Finally, I would like to thank my family for their continued support and encouragement during the course, specifically my mother Remani M P, my wife Sruthi Devi and my daughter Heyza.

\vfill
\clearpage
\section*{\centering Declaration}
% Taken from the MSc Thesis template, and edited for a PGT report
The material contained within this report has not previously been
submitted for a degree at the University of Birmingham or any other university.
The research reported within this report has been conducted by the author
unless indicated otherwise.\\
\\
\textbf{Signed} Sarathkumar Padinjare Marath Sankaranarayanan 

\vfill
\clearpage
\begin{center}
\vspace*{\fill}
\begin{minipage}{6in}

%\centering \Large{``Most men who have really lived have had, in some share, their great adventure.\\This railway is mine."}\\{\normalsize{\textsc{James J. Hill}}, \emph{Railway Pioneer}} \vspace{2cm}

%\centering \Large{``Steam engines don't answer back.\\ You can belt them with a hammer and they say nowt."}\\{\normalsize{\textsc{Fred Dibnah}}, \emph{Steeplejack and Engineer}} \vspace{2cm}

\centering \Large{``You have to learn the rules of the game.\\ And then you have to play better than anyone else"}\\{\normalsize{\textsc{Albert Einstein}}}

  \end{minipage}
  \vspace*{\fill}
\end{center}


\clearpage
\maketitle
\vspace{-5.5em} %fixes distance between \maketitle and the TOC
\begingroup
    \fontsize{9pt}{11pt}\selectfont
\tableofcontents
\endgroup
\clearpage
\phantomsection
\section*{Table of Abbreviations}

\normalsize
\begin{acronym}[SCEPTICS] % Give the longest label here so that the list is nicely aligned
\acro{3DES} {Triple DES}
\acro{3G} {Third Generation}
\end{acronym}

\addcontentsline{toc}{section}{Table of Abbreviations} 
\clearpage


\listoffigures

\addcontentsline{toc}{section}{List of Figures} 
\clearpage

\listoftables
\addcontentsline{toc}{section}{List of Tables} 
\clearpage

% set up the page numbering and counter - Table of Abbreviations has no page number
% also set up the footers and headers appropriately.
\pagenumbering{arabic}
\setcounter{page}{1}
\lhead{}\chead{MSc. Project Report :: \nouppercase{Section \thesection\minitab :: \leftmark}}\rhead{}
\rfoot{Page \thepage \hspace*{0.2pt} of \pageref{LastPage}}
\renewcommand{\headrulewidth}{0.4pt}

\section{Introduction}



This section will introduce the user to definitions of terms relevant for understanding the problem, discuss the motivation behind the problem, the aim \& approach taken to solve the problem, and the structure of this report. 


\subsection{Definitions}

\subsubsection{Credit Card Statement Date}
The credit card statement date is the date on which the statement/bill is generated every month. Any transaction conducted on the card post billing date will reflect in the next month's credit card statement.

\subsubsection{Delinquent Account}
A credit card account is considered delinquent if the customer has failed to make the minimum monthly payment for 30 days from the original due date.

\subsubsection{Delinquency Rate}
The percentage of credit card accounts within a financial institution's portfolio whose payments are delinquent.
\begin{equation}
	Delinquency Rate = \left(\frac{Number Of Delinquent Credit Card Accounts}{Total Number Of Credit Card Account}\right) * 100
\end{equation}

\subsubsection{Credit Card Default}
The customer is considered as defaulting customer in the event of nonpayment of the due amount in 120 days after the latest statement date.
\subsection{Overview}
Bank's consider credit cards as a gateway for the customer. Once the customer purchases the credit card from the bank, the customer will use the Bank's mobile app or web banking platforms to check balance, pay bills or other tasks. This provides the bank an opportunity to cross-sell other products such as Loan, Deposits, Remittances etc to the customer; thus, Credit Cards are an important part of Bank's product portfolio. From a customer's perspective, credit cards provide liquidity, convenience and security. Hence, credit cards are beneficial to both customer and the bank; due to this, the adoption of credit cards by the customers are increasing rapidly.\\

However, increased adoption of the credit card poses risk to the bank as the credit card default also increases along with the adoption. For example, just before 2008 recession the credit card adoption was all time high; despite this, according to the  figure \ref{fig:fredgraph}, the delinquency rates were at an all-time high just before the recession started in 2008. Bank's found it difficult to balance the books as more \& more customers started to default on the credit card payment after recession started.. After 2008 recession, strict regulations were introduced by the governments around the world to monitor the credit risk of the bank. One metric which is used to evaluate the risk is Delinquency rate of the bank, a low value for delinquency rate is preferred.\\

Predicting credit defaults is essential for managing risk in the consumer lending industry. Credit default prediction enables lenders to make the best possible lending decisions, improving customer satisfaction and fostering strong company economics. \\

\begin{figure}[ht]
	\centering
	\includegraphics[width=1.0\textwidth]{fredgraph}
	\caption[Delinquency rate on credit card loans for the period 1992-2022]{Delinquency rate on credit card loans for the period 1992-2022{\citep{fedgraph_delinquency_history}}.}
	\label{fig:fredgraph}
\end{figure}

\subsection{Motivation}

Existing models can be used to manage risk. However, developing models that perform better than those in use is feasible.

\subsection{Aim \& Approach}

This study aims to evaluate the performance of different \acf{ML} \&  \acf{DL} techniques on an industry scale dataset for credit card default prediction. Initial task is to find a credit card default prediction dataset which represents the current real world industry scenario. Then conduct a literature review and identify the most used \acs{ML} \& \acs{DL} techniques for credit card default prediction. Train the identified models on the dataset and compare the results; additionally, apply feature selection \& data oversampling/undersampling techniques and hyperparameter tuning to enhance the performance of the model. Validate the results with the understanding received from literature review. Finally, propose/suggest a model architecture for industry use cases to predict credit card default.

\subsection{Structure of Report}
The remainder of the report is structured as follows: in Section \ref{sec:background_knowledge} background information on different machine learning \& deep learning algorithms along with some data pre-processing techniques is provided. Then in Section \ref{sec:literature_review} a literature review related to the credit card default prediction research is given. Section \ref{sec:materials} \& \ref{sec:methodology} provides detailed explanations on the dataset, tools \& software used in the project, and methodology followed for creating the models. Model evaluation results and the comparison is given in Section \ref{sec:results_discussions}. Finally Section \ref{sec:conclusion} discusses the conclusion of the project. 
\vfill
\clearpage
\section{Background Knowledge} \label{sec:background_knowledge}
This section provides the reader with the required background information on the machine learning algorithms, deep learning architectures, data preprocessing techniques \& model evaluation metrics. The explanations provided in this section are at intutive level only without going deeper into the mathematical formulation.
\subsection{\acf{SVM}}
\acs{SVM} is a reliable classification and regression machine learning algorithm that maximizes the accuracy of a model without overfitting the training data. There are 4 main components to the \acs{SVM} model, Hyperplane, Support Vectors, Margin, Kernel function. Hyperplane refers to the decision boundary of the model, this could be a line if the data is 2 dimensional, plane if the data is 3 dimensional, or hyperplane if the data is n dimensional. Support Vectors refers to the data points that are nearest to the hyperplane and  these points are more difficult to classify. As shown in figure \ref{fig:svm}, the \acs{SVM} algorithm tries to find a hyperplane which maximizes the distance between the support vectors and the hyperplance and hence reducing the overfitting on the training set.
\begin{figure}[ht]
	\centering
	\includegraphics[width=0.4\textwidth]{svm}
	\caption[Support Vector Machine]{Components of \acs{SVM} \citep{rani2022machine}}
	\label{fig:svm}
\end{figure}
\acs{SVM} machinel learning technique uses a concept of Soft Margin to remove the issues that may arise due to outliers in the dataset.Additionally, if the decision boundary is non-linearly separable, \acs{SVM} transforms original data to map into new space using the Kernel function. Linear Kernel, Polynomial kernel \& \acf{RBF} are different kernel functions that can be used in \acs{SVM} model. There are multiple implementations of the \acs{SVM} technique is available; however, in this project libsvm\citep{chang2011libsvm} \& liblinear \citep{fan2008liblinear} is used as these implementations provide higher performance on large datasets.
\subsection{Ensemble Learning}
Ensemble methods are highly effective compared to the traditional machine learning techniques and considered state-of-the-art approach for solving many challenges \citep{sagi2018ensemble}. The idea behind ensemble learning technique is to train multiple base learners/models and combine the predictions from each learner to make the final prediction. As shown in figure \ref{fig:ensemble_learning}, ensemble techniques can be mainly categorized into 2 types Homogeneous \& Heterogeneous methods. In homogeneous methods all the base learners uses the same machine learning technique; on the other hand, heterogeneous methods may use different types machine learning techniques as the base learner.
\begin{figure}[ht]
	\centering
	\includegraphics[width=0.5\textwidth]{ensemble_learning}
	\caption[Ensemble Learning Techniques]{Ensemble Learning Techniques \& Implementations}
	\label{fig:ensemble_learning}
\end{figure}
\subsubsection{Bagging}
Bagging is a Homogeneous Ensemble Method where the base learners are trained in parallel on the complete training set or a subset of training set based on the configuration. As depicted in figure \ref{fig:bagging_boosting}, initially creates multiple dataset through random sampling with replacement, then train the mutiple learners in parallel. Finally combine output from all learners using either taking average or using majority vote strategy based on the problem is being solved. 
\begin{figure}[ht]
	\centering
	\includegraphics[width=0.5\textwidth]{bagging_boosting}
	\caption[Illustrations of bagging and boosting ensemble algorithms.]{Illustrations of bagging and boosting ensemble algorithms\citep{yang2019concepts}.}
	\label{fig:bagging_boosting}
\end{figure}
\subsubsection{Boosting}
Boosting is a Homogeneous Ensemble Method where the base learners are trained sequentially and the predictions from individual learners are combined to make the final prediction. The model starts by creating a base learner which performs slightly better than the random prediction, then subsequent learners try to improve on the prediction made by the previous learner. The process ends when the improvement made by the new learner is less than the threshold. As depicted in figure \ref{fig:bagging_boosting}, the boosting methods uses a weighted averaging strategy to combine the predictions from the individual learners. 
\subsubsection{Stacking}
Stacking is a Heterogeneous Ensemble Method where the output from the base learners is passed through another learner to make the final prediction. In Bagging \& Boosting methods the final prediction was made using taking average, majority voting, or weighted average; however the Stacking models uses a learner to make the final prediction. Figure \ref{fig:stacking} shows a sample Ensemble Stacking Model. Model A, B, and C are base learners which are trained parallely. The output from Model A, B \& C is then passed through a Generalizer/Meta Learner which predicts the final output. 
\begin{figure}[ht]
	\centering
	\includegraphics[width=0.5\textwidth]{stacking}
	\caption[An example scheme of stacking ensemble learning. ]{An example scheme of stacking ensemble learning \cite{divina2018stacking}.}
	\label{fig:stacking}
\end{figure}

\subsection{\acf{RF}}
\acf{RF} \citep{breiman2001random} is an ensemble bagging method technique which uses \acf{DT} to create base learners. \acs{DT} are a non-parametric supervised learning method used for classification and regression. The goal is to create a model that predicts the value of a target variable by learning simple decision rules inferred from the data features. \acs{RF} creates multiple \acs{DT} and trains each one of them with a random subsample of the dataset. The predictions from each \acs{DT} is then combined by either taking average or by using majority voting strategy.\\
\begin{figure}[ht]
	\centering
	\includegraphics[width=0.5\textwidth]{random_forest}
	\caption[An example \acs{RF}. ]{An example of \acf{RF} \cite{sapountzoglou2020fault}.}
	\label{fig:random_forest}
\end{figure}

\subsection{\acf{GBDT}}
Gradient-boosted decision trees are a machine learning technique for optimizing the predictive value of a model through successive steps in the learning process. Each iteration of the decision tree involves adjusting the values of the coefficients, weights, or biases applied to each of the input variables being used to predict the target value, with the goal of minimizing the loss function (the measure of difference between the predicted and actual target values). The gradient is the incremental adjustment made in each step of the process; boosting is a method of accelerating the improvement in predictive accuracy to a sufficiently optimum value.
\begin{figure}[ht]
	\centering
	\includegraphics[width=0.5\textwidth]{gbdt}
	\caption[Architecture of \acf{GBDT}]{Architecture of \acf{GBDT} \cite{deng2021ensemble}.}
	\label{fig:gbdt}
\end{figure}
\subsubsection{\acf{XGBoost} \& \acf{LGBM}} \label{sec:xgboost_lgbm}
\acf{XGBoost}\citep{chen2016xgboost} and \acf{LGBM}\citep{ke2017lightgbm} are two mostly used \acf{GBDT} framework/libraries which performs much better than the basic \acs{GBDT} models \citep{machado2019lightgbm}. \acf{XGBoost} \& \acs{LGBM} frameworks differ in how the individual weak learners are constructed. As shown in figure \ref{fig:lgbm_xgboost_tree_growth}, the \acs{XGBoost} uses Level Wise Tree Growth strategy while the \acs{LGBM} uses Leaf Wise Growth Strategy. \acf{GOSS} technique is used by the \acs{LGBM} to split the nodes while generating the individual weak learners; on the other hand, the \acs{XGBoost} uses Pre-Sorted and historgram based algorithms for splitting nodes.

\begin{figure}[ht]
	\centering
	\includegraphics[width=0.5\textwidth]{lgbm_xgboost_tree_growth}
	\caption[\acs{XGBoost} Level wise tree growth and \acs{LGBM} Leaf wise tree growth.]{\acs{XGBoost} Level wise tree growth and \acs{LGBM} Leaf wise tree growth.\cite{rezazadeh2020generalized}.}
	\label{fig:lgbm_xgboost_tree_growth}
\end{figure}

Both frameworks require the features to be numerical and does not support text features; thus; categorical encoding must be done before training \acs{LGBM} \& \acs{XGBoost} models.
In addition, both these libraries are well optimized for parallel processing and hence can be used for large datasets.

\subsection{\acf{ANN}}
\acs{ANN} are a subset of Machine Learning algorithms. The  architecture and name of \acs{ANN} is inspired by how the human brains works, specifically, how biological neurons signal to one another\citep{ibm2022neural}. Figure \ref{fig:ann} represents the architecture of a \acs{ANN} in general. The network consists of an input layer, multiple hidden layers and an output layer. There are multiple nodes, also known as neurons, in each layer. The neurons are the building blocks of the \acs{ANN} network. Each neuron connects to all the neurons in the next layer; moreover, neurons in the input layer are basically the input features.

\begin{figure}[ht]
	\centering
	\includegraphics[width=0.5\textwidth]{ann}
	\caption[Architecture of \acf{ANN}.]{Architecture of \acf{ANN}\cite{bre2018prediction}.}
	\label{fig:ann}
\end{figure}
Neurons are the basic building block of an \acs{ANN}. As shown in figure \ref{ann_node}, each neuron at least consists of a learning function and an activation function. Learning function can be thought of as a simple logistic/linear regression model and the activation function can be though of as a gatekeeper which decides how much influence this neuron should have in the next layer. Commonly used activation functions are Sigmoid, \acs{ReLU} and tanh. 
\begin{figure}[ht]
	\centering
	\includegraphics[width=0.5\textwidth]{ann_node}
	\caption[Representation of a Neuron in \acf{ANN}.]{Representation of a Neuron in \acf{ANN}.\cite{ghedira2004effect}}
	\label{fig:ann_node}
\end{figure}

While training \acs{ANN} models, a cost function is assigned to evaluate the performance of the model, \acf{MSE}, Binary Cross Entropy are some of the cost functions which are generally used. Finally the weights learning function of each neuron is adjusted through an algorithm called Backpropogation based on the result of the cost function. The algorithm ends when the cost function converges (figure \ref{fig:cost_function}) and the improvement made to the performance is less than the threshold.
\begin{figure}[ht]
	\centering
	\includegraphics[width=0.5\textwidth]{cost_function}
	\caption[Representation of Convergence of Cost Functions]{Representation of Convergence of Cost Functions\citep{ibm2022neural}}
	\label{fig:cost_function}
\end{figure}

\subsection{\acf{RNN}}
A \acf{RNN} is a type of \acs{ANN} which uses sequential data or time series data. \acs{RNN} has the capability of using previous inputs in the sequence to influence the current input and output, in other words, the \acs{RNN} has a memory of past events/data which it uses to calculate the current output.Figure \ref{fig:rnn_nn} shows comparison of a normal feed forward neural network (Tradition Neural Network)  and a \acs{RNN}. In a Feed Forward Neural Network the signals flow only in one direction from input to output; however, In an \acs{RNN} output of a layer is added to the next input and fed back into the same layer.Traditional Deep Learning networks assumes no dependency between the inputs and outputs, on the other hand, the output of \acs{RNN} depends on the previous inputs in the sequence.
\begin{figure}[ht]
	\centering
	\includegraphics[width=0.5\textwidth]{rnn_nn}
	\caption[Comparison of Recurrent Neural Networks (on the left) and Feedforward Neural Networks (on the right)]{Comparison of Recurrent Neural Networks (on the left) and Feedforward Neural Networks (on the right)\citep{ibm2022rnn}.}
	\label{fig:rnn_nn}
\end{figure}

Consider the Credit Card Default problem with 12 month data, the Rolled \acs{RNN} represents the entire Neural Network, and the Unrolled \acs{RNN} represents each layer in the Neural Network or each time step. In credit card default problem, X\textsubscript{0}  represents the data for the first month, X\textsubscript{1} for the second month and so on. \acs{RNN} uses \acf{BPTT} algorithm and gradient descent to learn the weights of the model while training.
\begin{figure}[ht]
	\centering
	\includegraphics[width=0.5\textwidth]{rnn_rolled_unrolled}
	\caption[\acs{RNN} rolled and unrolled]{\acs{RNN} rolled and unrolled \citep{ibm2022rnn}.}
	\label{fig:rnn_rolled_unrolled}
\end{figure}


There are 3 variants of the \acs{RNN} architectures as shown in figure \ref{fig:various_rnn}. While \acs{RNN} uses previous inputs to influence the current output, \acs{BRNN} use future inputs as well to improve the accuracy of the \acs{RNN} architecture. \acf{LSTM} is variant of \acs{RNN} proposed Sepp Hochreiter and Juergen Schmidhuber in their paper \citep{hochreiter1997long}. If the previous input which is influencing the current input is not in recent past, \acs{RNN} might not be able predict the current output accurately. \acs{LSTM} tries to solve this issue using cells in the hidden layers of the neural network. Cells have three different gates, input, output and forget gate, these gates control the flow of information which is needed to predict the output in the network. \acf{GRU} is similar to \acs{LSTM} and tries to resolve the issue of Short-term memory issue of \acs{RNN}. \acs{GRU} uses hidden states to regulate the information in the network and uses two gates, reset, update gate, to control the flow of information through network \citep{cho2014learning}.\\
\begin{figure}[ht]
	\centering
	\includegraphics[width=0.7\textwidth]{various_rnn}
	\caption[Variant \acs{RNN} Architectures]{Variant \acs{RNN} Architectures.}
	\label{fig:various_rnn}
\end{figure}

\subsection{\acf{GRU}}

\acf{GRU} is a variant of \acs{RNN} network which tries to solve the short-term memory issue of \acs{RNN} networks. Figure \ref{fig:gru_new} represents the structure for \acs{GRU} unit. Input to the current unit is the output hidden state from the previous input (h\textsubscript{t-1});furthermore, \acs{GRU} has two gates, update(z\textsubscript{t}) and reset(r\textsubscript{t}) gate. The algorithm uses reset gate to calculate the candidate output state (h\textsubscript{t}\textsuperscript{'}). Finally, update gate z\textsubscript{t}, previous hidden state h\textsubscript{t-1} and candidate output hidden state h\textsubscript{t}\textsuperscript{'} are used to calculate the final output hidden state h\textsubscript{t}.
\begin{figure}[ht]
	\centering
	\includegraphics[width=0.7\textwidth]{gru_new}
	\caption[\acs{GRU} Architecture]{\acs{GRU} Architecture \citep{simeon2017gru}}
	\label{fig:gru_new}
\end{figure}

Update gate is calculated using equation \ref{eq:gru_ugate} which is highlighted in figure \ref{fig:gru_detail} (a) of update gate. Update gate is formed adding the result of multiplication of  input x\textsubscript{t} with its own weight W\textsuperscript{z} and previous hidden state , h\textsubscript{t-1} with it's own weight U\textsuperscript{z}. Finally applying Sigmoid activation function to make the value between 0 and 1. Update gate determines how much of the information from the previous time steps needs to be passed to the future.
\begin{equation} \label{eq:gru_ugate}
	\centering
	\includegraphics[width=0.7\textwidth]{gru_ugate_eq}	
\end{equation}

Reset gate is calculated using equation \ref{eq:gru_rgate} which is highlighted in figure \ref{fig:gru_detail} (b). Reset gate may seem similar to update gate by looking at the formula;;however, the important thing to note is that the weights are different for update \& reset gate and hence it learns different characteristics of the network. Reset gate determines how much of the memory from past inputs to forget.

\begin{equation} \label{eq:gru_rgate}
	\centering
	\includegraphics[width=0.7\textwidth]{gru_reset_eq}	
\end{equation}

Calculation of new candidate output hidden state is shown in equation \ref{eq:gru_cstate} and \ref{fig:gru_detail} (c). Both the input and previous output hidden state is multiplied by the respective weights before taking sum. Additionally, the product of previous hidden state and the weight is further multiplied (Hadamard Product) with the reset gate, this operation determines how much of the previous information to be passed over to the current output hidden state.
\begin{equation} \label{eq:gru_cstate}
	\centering
	\includegraphics[width=0.7\textwidth]{gru_cstate_eq}	
\end{equation}

Finally the update gate, previous output hidden state and the candidate current output hidden state is used to calculate the final current output state as shown in equation \ref{eq:gru_nstate} and figure\ref{fig:gru_detail} (d). 
\begin{equation} \label{eq:gru_nstate}
	\centering
	\includegraphics[width=0.7\textwidth]{gru_nstate_eq}	
\end{equation}

\begin{figure}[h!]
	
	\begin{subfigure}{0.49 \textwidth}
		\includegraphics[width=1\linewidth, height=1\linewidth]{gru_ugate}
		\subcaption[Update Gate]{Update Gate}
		\label{fig:gru_ugate}
	\end{subfigure}
	\hfill
	\begin{subfigure}{0.49 \textwidth}
		\includegraphics[width=1\linewidth, height=1\linewidth]{gru_rgate}
		\caption[Reset Gate]{Reset Gate}
		\label{fig:gru_rgate}
	\end{subfigure}
	\begin{subfigure}{0.49 \textwidth}
		\includegraphics[width=1\linewidth, height=1\linewidth]{gru_cstate}
		\caption[Candidate Output State]{Candiate Output State}
		\label{fig:gru_cstate}
	\end{subfigure}
	\hfill
	\begin{subfigure}{0.49 \textwidth}
		\includegraphics[width=1\linewidth, height=1\linewidth]{gru_nstate}
		\caption[Final Output State]{Final Output State}
		\label{fig:gru_nstate}
	\end{subfigure}
	\caption[\acs{GRU} Architecture Stepwise]{\acs{GRU} Architecture Stepwise \citep{simeon2017gru}}
	\label{fig:gru_detail}
\end{figure}

\subsection{Feature Selection}
While building a Machine Learning model, all of the features might not contribute equivalently to the model's prediction performance, some of the features might impact the model's prediction performance adversely. This problem becomes more prominent on high dimensional data. The process of identifying the important features which improve the performance of the model is called Feature Selection. Two feature selection methods which were used in this project are Select From Model \& Sequential Feature Selection.
\subsubsection{Select From Model}
In this method, the training dataset is first trained on a machine learning model which is computationally less expensive and provides acceptable level of performance, this model can be considered as a filter. Then remove the features whose feature weights are less than the threshold value in the filter model. Finally, use the filtered dataset for training the main model. In this project, the \acs{SVM} is used as the filter model for training the main \acs{LGBM} model.
\subsubsection{Sequential Feature Selection}
As shown in figure \ref{fig:seq_feature_selection}, sequential feature selection starts with a subset of dataset and then an estimator chooses the best feature to add or remove to the dataset based on the \acf{CV} score obtained by adding/removing each feature. The process ends once the algorithm reaches the exit criteria. Exit criteria could be either the count of features or until improvement in performance gained by adding/removing features is not greater than a threshold.
\begin{figure}[ht]
	\centering
	\includegraphics[width=0.5\textwidth]{seq_feature_selection}
	\caption[Sequential Feature Selection Flowchart]{Sequential Feature Selection Process\citep{beyan2015detection}}
	\label{fig:seq_feature_selection}
\end{figure}
There are two types Sequential Feature Selection techniques, Forward Selection \& Backward Selection. In Forward Selection, the initial feature subset contains only one feature, then additional features are added based on \acs{CV} score until the exit criteria is met. On the other hand, in Backward Selection technique, the initial feature subset contains all the features, then the algorithm removes the features based on \acs{CV} score until the exit criteria is met. 
\subsection{Encoding Categorical Features}
As discussed in section \ref{sec:xgboost_lgbm}, some of the machine learning techniques does not support categorical text variables. Two most common methods used to encode categorical features are Ordinal Encoder \& One-hot encoder. Ordinal encoder is better than the One-hot encoder in memory efficiency.
\subsubsection{Ordinal Encoder}
In this method, the feature is transformed into a numerical value by assigning numbers to each distinct category. For eg: in Scikit Learn library the categorical variables low, medium, high will be transformed to 0, 1, 2; moreover, a default number can be assigned to categories which are not seen in the training dataset. 
\subsubsection{One-hot Encoder}
In this method, the categorical feature column is transformed into n different features each representing one distinct category. For eg: a feature X with low,medium,high as distinctive categorical values, will be transformed into 3 features X\_low, X\_medium, X\_high. X\_low will have a value of 1 if the X='low' for the record, likewise for the other features.
\subsection{Data Oversampling}
A classification data set with skewed class proportions is called imbalanced. Classes that make up a large proportion of the data set are called majority classes. Those that make up a smaller proportion are minority classes. Data sampling is a method used to overcome / reduce the effect of Class Imbalance on the performance of the model. There two types of Data Sampling, Over Sampling \& Under Sampling. Over Sampling techniques boost the minority class entries by introducing new records in minority class; on the other hand, the Under Sampling methods eliminates entries from majority class and makes the majority \& minority class entries to similar proportion.

In this project, Data Oversampling techniques are used as the dataset is large. \acf{SMOTE} \citep{chawla2002smote} \&  K-Means \acs{SMOTE} \citep{last2017oversampling} are two different Oversampling methods which were explored in this project.

\subsubsection{\acf{SMOTE}}
\acs{SMOTE} method introduces new data points in the minority class by finding nearest neighbors and adding new data point along the line of nearest neighbor. Figure \ref{fig:smote} shows the \acs{SMOTE} process, in this picture green points are minority class \&  blue points are majority class. The algorithm first selects a data point from the minority class, then finds the K nearest neighbors among the minority class. Finally one of the K nearest neighbor is choosen randomly and a new synthetic minority class data point (red) is added along the straight line connecting the selected data point and the choosen nearest neighbor.

\begin{figure}[ht]
	\centering
	\includegraphics[width=0.5\textwidth]{smote}
	\caption[\acs{SMOTE} algorithm]{\acs{SMOTE} algorithm \citep{schubachimbalance}}
	\label{fig:smote}
\end{figure}
\subsubsection{K-Means \acs{SMOTE}}
In K-Means \acs{SMOTE} method, the minority class is first passed through a K-Means clustering model before applying the \acs{SMOTE} algorithm. Clustering before applying \acs{SMOTE} helps to eliminate the problem of oversampling the outliers in the minority classes;moreover, clustering also helps to apply the \acs{SMOTE} algorithm for non-linearly separable data. As shown in  figure \ref{fig:kmeans_smote}, the SMOTE algorith is applied separately on each cluster.
\begin{figure}[ht]
	\centering
	\includegraphics[width=0.5\textwidth]{kmeans_smote}
	\caption[K-Means \acs{SMOTE} algorithm]{K-Means \acs{SMOTE} Flowchart \citep{chen2021research}}
	\label{fig:kmeans_smote}
\end{figure}

\subsection{\acf{CV}}

\acf{CV} is a model evaluation technique where a subset of the training dataset is removed and kept for evaluation and optimization of the model. Its a common practice to split the dataset into 3 set, Training, Validation \& Test set. Training set is used to build the model; on the other hand, Validation set is used to optimize the model parameters built on the Training set. Test set is used to evaluate the performance of the final model;importantly, it is assumed that the Test set has no influence on the model creation \& optimization and it is unseen while the model is trained and optimized.\\
\begin{figure}[ht]
	\centering
	\includegraphics[width=0.5\textwidth]{dataset_split}
	\caption[A sample Train,Validation \& Test Split]{A sample Train,Validation \& Test Split}
	\label{fig:dataset_split}
\end{figure}

Figure \ref{fig:cv_techniques} shows different Cross-Validation techniques, all these technique differs on how it selects/creates the validation set. For eg: Leave One Out Cross-Validation technique selects one data record from the Training set as the Validation set (without replacement) and the rest is used for training, this process is repeated n times ( number of data records in Traning set). The model evaluation metric collected in each iteration and the average of collected metric taken to get the final value of metric. Section \ref{sec:holdout} \& \ref{sec:kfold} will present couple of \acs{CV} techniques used in this project.\\
\begin{figure}[ht]
	\centering
	\includegraphics[width=0.7\textwidth]{cv_techniques}
	\caption[Different Cross-Validation Techniques]{Different Cross-Validation Techniques}
	\label{fig:cv_techniques}
\end{figure}

\subsubsection{Holdout \acs{CV}}\label{sec:holdout}
In this method, the Training set is split randomly into two set, one split is used to train the model and other is used as the Validation set to evaluate the model. Generally, the training split is larger than the Validation split. This method is commonly used if the dataset is large and balanced. This is because if the dataset is large, other cross-validation techniques will be more expensive to run;also, if the dataset is large, the model can generalize well using the training set itself and leaving out few entries for validation will not cause huge decline in performance.
\subsubsection{K-Fold \acs{CV}}\label{sec:kfold}
In this method, the Training set is split into K equal partitions, and then one partition is taken as validation set and the remaining partitions used to build the model. This process is repeated for every partition, ie, the process is repeated K times, each time taking a different partition as the validation set. This method is commonly used if the dataset size is moderate or to boost the performance of the model by training on the entire Training set. However, K-Fold \acs{CV} is computationally expensive as the model will be trained 5 times with different subset of data.
\subsection{Hyperparameter Tuning}
Every Machine Learning technique provides choices/parameters while creating model architecture to tune the algorithm to specific use cases. For eg: in \acf{RF} algorithm, the number of weak learners is a hyperparameter, a user can try different values for number of weak learners to see which value provides best performance for the model. Another example is number of layers in an \acf{ANN}, user can try different number of layers to tune the performance of the model. The process of trying different values for hyperparameters to find the best model architecture is known as Hyperparameter Tuning. The entire Training set, including the validation set, is trained on the model using best hyperparameters found using Hyperaparameter Tuning process to get the final model. Figure \ref{fig:hyperparameter_tuning_methods} shows different techniques used for the Hyperparameter tuning process.
\begin{figure}[ht]
	\centering
	\includegraphics[width=0.5\textwidth]{hyperparameter_tuning_methods}
	\caption[A sample Train,Validation \& Test Split]{A sample Train,Validation \& Test Split}
	\label{fig:hyperparameter_tuning_methods}
\end{figure}


\subsubsection{Grid Search} \label{sec:grid_search}
In this method, the model is trained using each possible combination of hyperparameters and the best model is the one which scores the highest value for the chosen metrics on the Validation set. The choices for hyperparameters are discrete; for example, X \& Y  are two hyperparameters for the model Z, choices for X = [1, 2, 3] \& Y=[4, 5, 6], then Grid Search algorithm will train the model on every possible combination of X \& Y and will find the best combination of X \& Y which provides the highest performance. Grid Search \acs{CV} is an API in Scikit Learn Libray which uses K-Fold Validation technique to assess the performance of the model while performing Grid Search algorithm for hyperparameter tuning.

\subsection{File Format - Parquet}
The file format of the dataset has a significant role in the overall model building process. The Primary Dataset used in this project was initially hosted in Google Drive (Cloud Storage) as a csv file and the dataset size was 16GB. Transferring the dataset into Google Collab runtime environment took lot of time and effected overall model building time. Due to this, each iteration  of the experiment took more time which was impacting overall progress of the project. In order to overcome this issue, the dataset was converted to Parquet  file format which helped to reduced the dataset size to 7GB.

In 2013, Twitter \& Cloudera announced a new open-source columnar storage format, Parquet. Instead of storing the millions of records in row wise, the Parquet format stores the data column wise. Due to this, the Parquet format is able to compress the data better than the other file formats. Moreover, the data retrieval in columnar storage is also more efficient for the analytical usecases. 


\subsection{Summary}
In this section, intuitive level explanation of all the techniques and algorithms used in this project was provided. The next section will discuss some of the previous studies conducted on Credit Card Default Prediction task.
\vfill
\clearpage
\section{Literature Review}\label{sec:literature_review}
\subsection{Introduction}
This section discusses the current techniques used to predict the credit card default. Since the dataset used for these studies differ, a direct comparison of results is not possible. However, an overall comparison of different techniques and its efficiency in predicting the credit card default will be discussed wherever possible.

\subsection{Previous Work}
\citep{sayjadah2018credit} developed logistic regression, rpart decision tree \& Random Forest Classifier models on Taiwan Clients Dataset \citep{yeh2009comparisons} Dataset. The dataset contains 30000 records and 24 features generated from credit card transactions. A \acf{CFS} technique was used to reduce the dimensionality of the dataset and 30\% of the dataset was used as the test set for evaluating the performance. \citep{sayjadah2018credit} found that the Random Forest Classifier scored highest \acf{AUC} metric among the all models.\\

\citep{widyadhanacredit} developed Logistic Regressiion, \acs{SVM}, \acs{ANN} \& Random Forest Classifier models to preduct credit card default on dataset containing 1000 records and 11 features. The dataset contains credit card data from cardholders from the territory of Indonesia;moreover, the authors used \acf{PCA} to do feature selection also. The dataset is split into 70:30 Train/Test set and the \acs{AUC} metric was used to compare the results. The authors found that the Random Forest Classifier outperformed all other models by far and provided a 80\% \acs{AUC} score.\\


\citep{alam2020investigation} investigated different approaches to solve the credit card default prediction problem with a specific focus on class imbalance issue. The study experimented the models on 3 different dataset, Taiwan Clients Dataset (30000 records) \citep{yeh2009comparisons}, Sought German Clients Credit Dataset(1000 Records) \& Belgium Clients Credit Dataset (300000 records).  The credit card default prediction dataset are inherently imbalanced as only a small fraction of customers default on credit card. The authors employed different data Under/Over sampling techniques and evaluated performance on multiple credit card default dataset. They found that \acs{GBDT} classifier when used with K-Means \acs{SMOTE} provided the best results and the models performed significantly better on balanced datasets compared to imbalanced datasets.\\

\citep{faraj2021comparison} research shows that ensemble  methods  consistently  outperform  Neural  Networks  and  other  machine  learning algorithms in terms of F1 score. \citep{faraj2021comparison} uses the same dataset as the \citep{sayjadah2018credit} which has 30,000 records and 24 features. The authors found that \acs{XGBoost} provided maximum F1 score compared to Neural Networks, Random Forest Classifier and custom ensemble stacking model. Authors also concludes that the performance of \acs{XGBoost} model did not improve on balanced datasets. \citep{emil2019enhancing} also compared the ensemble bagging,boosting techniques on the same data used by \citep{faraj2021comparison} and found that 
the Ensemble Boosting techniques provided highest predictive accuracy.  \\

Study conducted by \cite{yang2018comparison} compares the performance of the \acs{SVM}, \acs{ANN}, \acs{XGBoost}, and \acs{LGBM} algorithms on the Taiwan Credit Card dataset. The authors used 10-Fold cross-validation to optimze the model and  F1-Score metric was used to compare the performance of the models. Similar to the other studies \citep{faraj2021comparison} \citep{emil2019enhancing}, the authors found that the Ensemble Boosting techniques provided the highest performance on the dataset. Among the Ensemble Boosting frameworks, \acs{LGBM} provided the best F1-Score.\\

\citep{hsu2019enhanced} approached the credit card default prediction from a different perspective and proposed a model where dynamic features (time dependent features) were first passed through a \acf{RNN} network to extract the time dependent features. Then the extracted dynamic features were concatenated with the static features and trained on a Random Forest Classifer. The dataset contained 30,000 samples credit card payment history with 23 features (5 static feature, 18 dynamic features)\citep{yeh2009comparisons}. The authors compared the results of the proposed model with the \acs{SVM}, Logistic Regression and \acs{KNN} models and found that proposed method outperformed the others and provided a \acs{AUC} score of 78\%.\\

\citep{gao2021research} proposed \acs{XGBoost}-\acs{LSTM} method to predict the credit card default. In this approach, \acs{LSTM} is used to embed the time series transaction data, which is then combined with the basic data of the customer before training the same using \acs{XGBoost} algorithm. The authors identified that the \acs{XGBoost}-\acs{LSTM} performed better than the \acs{XGBoost} technique on the Recall metric. The dataset used for this study was obtained from a small-scale commercial bank \citep{gao2021research}.\\

\cite{chen2021research} proposed \acs{ANN}-KMeans \acs{SMOTE} technique to predict the credit card default. This method uses the K-Means \acs{SMOTE} algorithm to solve the Dataset Imbalance issue and uses \acs{RF} to extract the feature importance. The feature importances are then substituted to the initial weights of the \acs{ANN} network. The authors compared \acs{SVM}, \acs{DT} and \acs{RF} with the proposed model and found that the proposed model performed slightly better than the \acs{SVM} model on F1-Score. Authors also observes that the increase in performance gained by substituting initial weights with the feature importance obtained from \acs{RF} is not obvious.\\

A recent study \citep{wu2022cdgat} developed a method called \acf{CDGAT}. The dataset used for this study was obtained from  Industrial and Commercial Bank of China (Macau) Limited (ICBC (Macau)) and contains transaction data between 2015 to 2019. The authors used \acs{RNN} to encode the customer transaction  and Amount bias Sampling technique to obtain the neighborhood embedding . The proposed \acs{CDGAT} method was compared with the Logistic Regression, \acs{SVM}, \acs{LGBM} and \acs{ANN}  models on \acs{AUC} score. \acs{CDGAT} method performed significantly better than the all other models. \\

\citep{lawi2018classification} conducted a study comparing performance of the models \acs{SVM},\acf{LS-SVM} \citep{suykens1999least}  and an ensemble model based on \acs{LS-SVM}  on Taiwan Clients Dataset (30000 records) \citep{yeh2009comparisons}. The results shows that the \acs{LS-SVM} performed better on Sensitivity metric than the \acs{SVM} model. The \acs{LS-SVM} Ensemble did not improve the Sensitivity of the prediction from the \acs{LS-SVM} model; however, the \acs{LS-SVM} model provided the best score on Specificity. \\

\subsection{Summary}
In conclusion, the ensemble boosting models generally provided better performance than the classic machine learning and deep learning techniques on Credit Card Default prediction problem. The data Under/Over sampling had mixed impact on the predictive accuracy depending on the dataset, some models performed really better with Under/Over sampling and some did not. Out of all the data Under/Over sampling techniques, K-Means \acs{SMOTE} performed better. Some studies used feature selection techniques in the data pre-processing pipeline which improved the model efficiency. Most of the studies on the credit card default problem uses Taiwan Clients Dataset (30000 records) \citep{yeh2009comparisons} which only contains 30,000 records and the data is from period 2004-2005; however, this dataset does not completely represent today's industry scenario and the challenges where banks have millions of records which can aid the credit card default prediction. Also the consumer behavior might have changed over the time;hence, study of Credit Card Default problem on large scale \& recent data has significance in the current landscape. American Express Default Predition  dataset \citep{amex-default-prediction-dataset} contains 5 Million records of 400,000 unique customers from the period 2018-19, and exploring the different techniques on such large scale dataset will help to consolidate the understanding gained from these papers.

\vfill
\clearpage
\section{Materials}\label{sec:materials}

\subsection{Primary Dataset}
The primary dataset contains 190 aggregated profile features of 458913 American Express customers at each statement date for 13 months. Features are anonymized and normalized, and fall into the following general categories:

\begin{itemize}
	\item D\_* = Delinquency variables
	\item S\_* = Spend variables
	\item P\_* = Payment variables
	\item B\_* = Balance variables
	\item R\_* = Risk variables	
\end{itemize}

This dataset\citep{amex-default-prediction-dataset} was released as part of the "American Express - Default Prediction" hosted in Kaggle by the American Express team.

\subsection{Secondary Dataset}
The secondary dataset was derived from primary dataset by applying the below mathematical aggregate operations to the numerical features.
\begin{itemize}
	\item Minimum 
	\item Maximum
	\item Mean
	\item Last Value
	\item Standard Deviation
\end{itemize}

Aggregate for the categorical features were taken by the applying below operations.
\begin{itemize}
	\item Last Value
	\item Count
	\item Unique Value Count
\end{itemize}

The secondary dataset contains 920 features and 458913 records.

\subsection{Tools \& Software}
The primary programming language used for the implementation of this project is Python version 3.7. Data analysis and manipulation is done using Pandas(1.3.5), seaborn(0.11.2) \& Dask(2.12.0) packages. Scikit Learn(1.0.2) package is used for create, train \& evaluate machine learning models. \acs{ANN} \& \acs{GRU} models were created using Tensorflow (2.8.2).Google colab was used to train the model in cloud and Github was used as the version control \& project management software.

\vfill
\clearpage
\section{Methodology}\label{sec:methodology}
\subsection{Introduction}
This section will first provide a brief overview of the overall strategy of the experiments performed as part of this project followed by  providing detailed explanation on the data preprocessing techniques used. Then in subsequent sections each experiment/model will be presented along with the model specific explanations \& details.
\vfill
\clearpage
\subsection{Overview of Methodology Followed}
Figure \ref{fig:methodology} represents a overview of methodology in general followed for conducting experiments. The dataset was first split into chunks and stored in different files in parquet format to optimize the memory usage. Then, dataset was preprocessed to remove invalid values and encode categorical text variables to numerical values. Followed by data pre-processing, the dataset was split into Training \& Test set, this ensures that none of the entries in test set will have an influence in model training and model selection process. Then the dataset was enhanced using oversampling techniques to resolve the class imbalance issue;in addition, feature selection techniques were used to eliminate the features from the dataset which were less important and hence contribute very little to model. 
\begin{figure}[ht]
	\centering
	\includegraphics[width=0.8\textwidth, height=0.5\textheight]{methodology}
	\caption[Methodology]{Methodology followed for the experiments}
	\label{fig:methodology}
\end{figure}

After the feature selection, the model was created and then passed through a Hyperparameter tuning pipeline which helps to find the best parameters for the model which would give highest cross validation score. Finally the entire training dataset was trained on the best model found using hyperparameter tuning and the model was evaluated using the test set set aside at the beginning of the experiment.

\subsection{Data Preprocessing} 
This section discusses the common preprocessing techniques used in all experiments conducted as part of the project. The model specific data preporcessing techniques used will be discussed in respective sections describing the model.

\subsubsection{Default Values}
NaN \& NULL values in the dataset was replaced by Zero and if a column contains all values same, it was removed from the dataset. -1 was used as the default value for the categorical variables. The categorical variables were encoded using Ordinal Encoder before passing to the model training pipeline.

\subsubsection{Normalization}
The primary dataset from American Express is already normalized and all the values lies between zero to ten, hence none of the data normalization techniques were used to preporcess the data.
\subsubsection{Handling Memory Issue}
Google Colab provides 24 GB of \acf{RAM} in the  virtual environment, though the Primary Dataset is 16 GB,  the pandas library was unable to load the complete data into memory due to memory leakage issue in the framework. Dask library, which uses multiple Pandas dataframe under the hood, was used to overcome the memory. Dataframe API in Dask library splits the dataset into multiple chunks and loads each chunk on a need basis only ( Lazy Loading), this ensured that the complete 16 GB dataset could be loaded even at a low memory of 4GB.

Additionally, the primary dataset was loaded using Dask framework and split the dataset month wise, ie one file for each month. The month wise files were saved in parquet format which helped to reduce the total size of the dataset from 16GB to 7GB. Similarly the primary dataset was also split customer wise, ie 1-50000 customers data in one file, 50001-100000 customers data in second file etc. These files were later used to build the Secondary Dataset.

\subsection{Model 1 - \acf{SVM}}
The \acs{SVM} model was created with parameters Regularziation Term = L2 Norm(Squared Error Loss), Alpha = 0.0001, Loss='hinge'(soft-margin), tolerance=0.001. The primary dataset was used to train the model and the model converged after 28 iterations. Early stopping was used to prevent overfitting of the model and 10\% of the data from training set used as the validation set. Stochastic gradient descent was used to optimize the objective function, this ensured that even though the dataset contains millions of records, the training is able to proceed and finish in reasonable time.  20\% of the Secondary Dataset was used as test set to evaluate the performance of the model.
\subsection{Model 2 - Random Forest Classifier}
The Random Forest Classifier model uses 100 Decision Trees trained in parallel on the primary dataset. Each decision tree uses a different subset of Primary Dataset with maximum number of records in a database set to 600,000. Gini impurity metric is used to measure the quality of the split while building decision tree. Finally the model predicts the target variable by taking mean of all the predictions from the 100 individual decision trees. 20\% of the Secondary Dataset was used as test set to evaluate the performance of the model.
\subsection{Model 3 - \acf{GBDT}}
\acs{GBDT} model was created using 100 Decision Trees trained sequentially on the primary dataset. Friedman \acf{MSE} is used to measure the quality of a split; additionally, model was set to use only 60\% of the data for constructing each decision trees to avoid memory leakage issue. 10\% of the training set was set for validation purpose; furthermore, the parameters were set to stop the training if the validation score does not improve to avoid overfitting. Loss function for the training was set to Deviance. 20\% of the Secondary Dataset was used as test set to evaluate the performance of the model.
\subsection{Model 4 - \acf{XGBoost}}
\acs{XGBoost} model was created using training 100 base learners on the Secondary Dataset and each base learner is constructed using 80\% of the training dataset. Instead of using complete features to construct the base learner, parameters were set to use only 60\% of features, this helped to eliminate the memory leakage/overflow issues while training. Moreover, L2 regularization parameter was set to 0.9 to reduce the overfitting of the model. 20\% of the Secondary Dataset was used as test set to evaluate the performance of the model.

\subsection{Model 5 - \acf{LGBM}}
\acs{LGBM} model was trained on Secondary dataset and the 100 base learners were constructed using the entire features \& training set. Tradition Gradient Boosting Decision Trees were used as the boosting type and learning rate was set to 0.1. 20\% of the dataset were set aside as the test set for evaluating the model. Maximum depth is not set as to allow trees of any depth. 

\subsection{Model 6 - \acf{ANN}}
Figure \ref{fig:nn_arch} depicts the architecture of the custom \acs{ANN} model developed.  The primary dataset was first split into training \& test dataset, followed by oversampling the training dataset using K-Means \acs{SMOTE} to make the percentage of defaulting \& non defaulting customers equal. Then the training dataset was trained using the custom \acs{ANN} model. The first \& second layer uses \acf{ReLU} as the activation function, however the final layer uses Sigmoid as the activation function. Adam optimizer was used to optimze the objective binary cross entropy loss function. The trained model was tested and evaluated on the test set.\\
\begin{figure}[ht]
	\centering
	\includegraphics[width=0.6\textwidth, height=0.3\textheight]{nn_arch}
	\caption[Custom Neural Network Architecture]{Custom Neural Network Architecture}
	\label{fig:nn_arch}
\end{figure}

\subsection{Model 7 - \acf{GRU}}
Figure \ref{fig:gru_arch} represents the architecture of the GRU based model for predicting credit card default. The primary dataset is used for training this model;furthermore, the tanh function is used as the activation function and sigmoid  is used as the recurrent activation function  for the GRU layer. Dropout of 10\%, recurrent dropout of 50\% added to reduce the overfitting problem. The output of the GRU layer is then fed to dense layer followed by another dense layer with activation sigmoid for making final prediction. Optimizer used is Adam \& the loss function is Binary Cross Entropy. A dataset generator was created to provide input to the model in chunks, this helped to eliminate the memory issues while training. The final model contains 49165 trainable parameters.\\

\begin{figure}[ht]
	\centering
	\includegraphics[width=0.6\textwidth, height=0.3\textheight]{gru_arch}
	\caption[GRU Model Architecture]{GRU Model Architecture\\}
	\label{fig:gru_arch}
\end{figure}
\subsection{Model 8 - Ensemble Stacking Model using \acs{GRU}+\acs{ANN}+\acs{GBDT}}
Figure \ref{fig:gru_nn_gbdt_arch} partially depicts the architecture of the custom ensemble stacking model. The primary dataset is first trained using \acs{GRU} layer followed by a dense layer. In parallel, the secondary dataset is trained using a 2 dense layers. Then the output of these two parallel legs were combined to form the concatenation layer. The output of concatenation layer is then trained using a  \acs{GBDT} model to get the final prediction. Adam optimizer was used to optimze the objective binary cross entropy loss function. Instead of loading complete dataset into memory and train the entire dataset in one go, a dataset generator was written to return chunks of data for training, this helped to eliminate the memory overflow issues.\\
\begin{figure}[ht]
	\centering
	\includegraphics[width=0.7\textwidth, height=0.3\textheight]{GRU_NN_GBDT_Architecture}
	\caption[Custom Ensemble Stacking Model Architecture]{Custom Ensemble Stacking Model Architecture}
	\label{fig:gru_nn_gbdt_arch}
\end{figure}
\subsection{Model 9 - Lean \acs{LGBM} Model}
Finally, a lean model was created using \acs{LGBM} model which peformed on par with the other models but with less resources \& data. Firstly the 20\% of the dataset was set aside as test set and remaining 80\% for training purpose. Training dataset was then oversampled using K-Means \acs{SMOTE} method which resulted in the number of defaulting customers \& non defaulting customers to become equal. Secondly the the training dataset was trained using a simple \acs{SVM} model to extract the feature importances; in addition, the features with feature importance weight less than the mean of importance of weights were discarded. This helped to reduced the feature count from 920 in the secondary dataset to 279. This modified dataset was then passed through a Grid Search CV pipeline to choose the best parameters for maximum depth \& maximum number of leafs for base learner trees, and boosting type. \acf{CV} Recall score was used as the metric to choose the best model. Finally, a \acs{LGBM} model was created using the best model parameters found using GridSearchCV and trained the same on the complete training dataset. The final model was tested and evaluated on the test set.

\subsection{Summary}
Firstly, this section presented overall methodology followed for the experiments followed by the data preprocessing techniques used. Handling of invalid feature values, normalization \& handling memory leakage and memory overflow issue were discussed in the Data Preprocessing section. Secondly, the 9 different models created to solve the problem were discussed. Initially \acs{SVM} model \& Random Forest Classifier model were discussed followed by more advanced machine learning techniques such as \acs{GBDT}, \acs{XGBoost} \& \acs{LGBM}. Then 3 models using deep learning techniques such as Neural Network, \acs{GRU}  and Ensemble Stacking model were presented. Finally a lean model was developed and presented which used less features for training, was computationally less expensive and was explainable model. Before training the final model, the model was passed through a data oversampling pipeline, feature selection pipeline. The model was also tuned using GridSearchCV method to find the optimal hyper parameters. In all the experiments 20\% of data was set aside before training for testing \& evaluation purposes.
\vfill
\clearpage
\section{Results \& Discussions}\label{sec:results_discussions}

\subsection{Introduction}
Firstly this section presents the Metrics used to evaluate the model and the test result of the experiments conducted based on the methodology described in section \ref{sec:methodology}. Secondly the main highlights or achievements of the project is discussed. Finally, limitations of this project as well as the direction for the future works are reviewed.\\
\subsection{Metrics}
Figure \ref{fig:confusion_matrix} represents the Confusion Matrix associated with binary classification problem. There are 4 important terms related to confusion matrix, \acf{TP}, \acf{TN}, \acf{FP}, and \acf{FN}. \acs{TP} represents the predictions where the truth label and the predicted labels are both positive; on the other hand, \acs{FP} represents the predictions where the  predicted label is positive but the truth label is negative. \acs{TN} represents the predictions where both the predicted label and the truth label are Negative. Predictions where the predicted label is negative and the truth label is positive is called \acs{FN}.\\
\begin{figure}[ht]
	\centering
	\includegraphics[width=0.5\textwidth]{confusion_matrix}
	\caption[Confusion Matrix of A Binary Classification Problem]{Confusion Matrix of A Binary Classification Problem\citep{confusion_matrix}}
	\label{fig:confusion_matrix}
\end{figure}

There are multiple metrics which can be used to evaluate the performance of the model based on the confusion matrix, metrics which are relevant to this project is discussed in below sections.
\subsubsection{Accuracy}
Accuracy is defined as the ratio of total number of correct predictions and total number of samples. However, the Accuracy metric might be misleading on imbalanced dataset;for example, in a dataset where the minority class is only one percent of the overall dataset, 99\% accuracy can be achieved by predicting every data point as the majority class. Equation \ref{eq:accuracy} shows the formula for calculating Accuracy.
\begin{equation} \label{eq:accuracy}
	Accuracy = \frac{TP+TN}{TP+TN+FP+FN}
\end{equation}
\subsubsection{Recall}
Recall represents the ratio of total number of positive predictions and the total number of positive samples. Intuitively, Recall provides an idea of how many positive samples where correctly identified by the model;for example, in a credit card default prediction dataset, recall is how many of the defaulted customers were identified correctly by the model. Equation \ref{eq:recall} shows the formula for calculating Recall.
\begin{equation}\label{eq:recall}
	Recall = \frac{TP}{TP+FN}
\end{equation}
\subsubsection{Precision}
Precision is defined as the ratio of correct positive predictions and the total number of positive predictions. Precision provides insight into how many of the predicted positive samples where actually correct.Equation \label{eq:precision} shows the formula for calculating Precision.
\begin{equation}\label{eq:precision}
	Precision = \frac{TP}{TP+FP}
\end{equation}

\subsubsection{F1-Score}
F1-Score combines both Precision \& Recall into a single metric; in other words, F1-Score is the harmonic mean of Precision and Recall metrics. F1-Score has shown to be a good metric on the imbalanced datasets classification problems. F1-Score better represents the performance of the model on the dataset compared to the other metrics. Equation \ref{eq:f1score} shows the formula for calculating the F1-Score.
\begin{equation}\label{eq:f1score}
	Precision = 2 * \frac{Precision*Recall}{Precision + Recall}
\end{equation}

\subsection{Test Results}
Table \ref{table:results} depicts the result of running the models discussed in section \ref{sec:methodology} on the test set. 4 metrics, F1 Score, Recall, Accuracy \& Precision is used to compare the performance of the models. Though 4 metrics are tracked, Recall \& F1 Score are the metric which is used for Hyper Parameter Tuning, Early Stopping and Cross Validation across all experiments. This decision was taken based on observation that Accuracy will not be a good metric as the dataset is imbalanced; furthermore, the F1 score \& Recall provides a better indictation on how well the model is able to predict the defaulting clients correctly.

\begin{table}[h]
	\begin{center}
		\begin{tabular}{|| c | c | c | c | c ||} 
			\hline
			Model & F1 Score & Recall & Accuracy & Precision \\ [0.5ex] 
			\hline\hline
			SVM	 & 74.66	& 75.57	& 87.24	& \cellcolor[HTML]{ff6666} 73.78 \\
			\hline
			RF	 & \cellcolor[HTML]{ff6666} 73.81	& \cellcolor[HTML]{ff6666} 72.78	& \cellcolor[HTML]{ff6666} 87.13	& 74.86 \\
			\hline
			GBDT	 & 74.33	& 74.13	& 87.25	& 74.52 \\
			\hline
			XGBoost	 & 80.22	& 80.11	& 89.79	& 80.34 \\
			\hline
			LGBM	 & 81.01	& 81.11	& 90.13	& 80.91 \\
			\hline
			ANN	 & \cellcolor[HTML]{339933} 81.03	&  \cellcolor[HTML]{339933} 81.81	& 90.09	& 80.27 \\
			\hline
			GRU	 & 79.81	& 79.96	& 90.67	& 79.65 \\
			\hline
			GRU+ANN+GBDT	 & 80.09	& 79.71	& \cellcolor[HTML]{339933} 90.87	& 80.49 \\
			\hline
			Lean LGBM	 & 80.95	& 80.68	& 90.15	& \cellcolor[HTML]{339933} 81.23 \\
			\hline
		\end{tabular}
		\caption{Performance Comparison of Various Models.}
		\label{table:results}
	\end{center}
\end{table}

The \acs{SVM} provided an accuracy of 87.24\% \& F1 score of 74.66 which was impressive considering the model took very less time to train compared to other models; thus, \acs{SVM} model was used in Lean \acs{LGBM} model as part of feature selection pipeline. \acs{RF} model performed worst on F1 Score, Recall \& Accuracy among all the models experimented as part of this project. This reduced performance of \acs{RF} could be because the algorithm is not able to generalize well with 100 decision trees on a large dataset with high dimensionality. \acs{GBDT} model provided similar results to \acs{SVM} model, however, \acs{GBDT} provided better Precision while \acs{SVM} provided better Recall scores. Ensemble boosting techniques \acs{XGBoost} \& \acs{LGBM} performed really well on the test set such that F1 score \& Recall increased almost by  6\% compared to \acs{SVM} model. \acs{LGBM} model performed slightly better than \acs{XGBoost} model in all the metrics and \acs{LGBM} model took less time to train compared to \acs{XGBoost}. The improved training time of \acs{LGBM} compared to the \acs{XGBoost} is because the \acs{LGBM} uses histogram based techniques.\\

\acs{ANN} model provided the highest F1 Score \& Recall among all the models, however, the difference in F1 Score of \acs{LGBM} and \acs{ANN} model is only 0.02. \acs{ANN}  model provided better Recall score while \acs{LGBM} provided better Precision score. \acs{GRU}+\acs{ANN}+\acs{GBDT} model provided best accuracy score of 90.87 among all the models. \acs{GRU} provided the second best accuracy score;moreover, considering that \acs{GRU} model has 49,165 trainable parameters while the \acs{GRU}+\acs{ANN}+\acs{GBDT} has 178,945, the \acs{GRU} model is able to generalize better than \acs{GRU}+\acs{ANN}+\acs{GBDT} model with less parameters. The table \ref{table:trainable_params} shows the trainable parameter count for the deep learning models explored.

\begin{table}[h]
	\begin{center}
		\begin{tabular}{|| c | c ||} 
			\hline
			Model & Trainable Parameters \\ [0.5ex] 
			\hline\hline
			ANN	& 125,953 \\
			\hline
			GRU	& 49,165 \\
			\hline
			GRU+ANN+GBDT	& 178,945 \\
			\hline
		\end{tabular}
		\caption{Trainable parameters comparison of Deep Learning Models}
		\label{table:trainable_params}
	\end{center}
\end{table}

The proposed Lean \acs{LGBM} model provides a F1 Score of 80.95 which is only 0.08 less than the best performing \acs{ANN} model; furthermore, the Lean \acs{LGBM} provided the best Precision score among all the models. Lean \acs{LGBM} model uses only 1/3rd of the features used by the \acs{ANN} model and still provides comparable performance. The figure \ref{fig:cm_ml} represents the confusion matrix of all the Machine Learning models explored during this project.

\begin{figure}[h!]
	
	\begin{subfigure}{0.4 \textwidth}
		\includegraphics[width=1\linewidth, height=0.8\linewidth]{cm_svm}
		\subcaption[Support Vector Machine]{\acs{SVM}}
		\label{fig:cm_svm}
	\end{subfigure}
	\hfill
	\begin{subfigure}{0.4 \textwidth}
		\includegraphics[width=1\linewidth, height=0.8\linewidth]{cm_rf}
		\caption[Random Forest]{\acs{RF}}
		\label{fig:cm_rf}
	\end{subfigure}
	\begin{subfigure}{0.4 \textwidth}
		\includegraphics[width=1\linewidth, height=0.8\linewidth]{cm_gbdt}
		\caption[Gradient Boosting Decision Tree]{\acs{GBDT}}
		\label{fig:cm_gbdt}
	\end{subfigure}
	\hfill
	\begin{subfigure}{0.4 \textwidth}
		\includegraphics[width=1\linewidth, height=0.8\linewidth]{cm_xgboost}
		\caption[Xtreme Gradient Boosting Descision Tree]{\acs{XGBoost}}
		\label{fig:cm_xgboost}
	\end{subfigure}
	\begin{subfigure}{0.4 \textwidth}
		\includegraphics[width=1\linewidth, height=0.8\linewidth]{cm_lgbm}
		\caption[Light Gradient Boosting Machine]{\acs{LGBM}}
		\label{fig:cm_lgbm}
	\end{subfigure}
	\hfill
		\begin{subfigure}{0.4 \textwidth}
		\includegraphics[width=1\linewidth, height=0.8\linewidth]{cm_lean_lgbm}
		\caption[Lean Light Gradient Boosting Machine]{Lean LGBM}
		\label{fig:cm_lean_lgbm}
	\end{subfigure}
	\caption[Confusion Matrix of Models]{Confusion Matrix Models}
	\label{fig:cm_ml}
\end{figure}
\subsubsection{Feature Importance} \label{subsec:feature_imp}
The table \ref{table:imp_features} represents the features that contribute most to the predictive accuracy of the Lean \acs{LGBM} model. Interpretability  of the Lean \acs{LGBM} model is an another advantage over the \acs{ANN} model. However, since the features in the Primary \& Secondary Dataset is anonymized, it is not possible to know the exact real world meaning of these features in this study.

\begin{table}[h]
	\begin{center}
		\begin{tabular}{|| c | c ||} 
			\hline
			Feature & Importance Score \\ [0.5ex] 
			\hline\hline
			D\_39\_last &	226 \\
			\hline
			P\_2\_last &	214 \\
			\hline
			B\_4\_last &	191 \\
			\hline
			B\_3\_last &	162 \\
			\hline
			B\_1\_last &	144 \\
			\hline
		\end{tabular}
		\caption{Feature Importance Based on Lean \acs{LGBM} model  (Secondary Dataset)}
		\label{table:imp_features}
	\end{center}
\end{table}


\subsection{Achievements}
As mentioned in \ref{sec:literature_review}, most of the previous work on credit card default prediction was based on dataset containing records less than 300,000. However the Primary Dataset used in this work has around 5 Million records of 400,000 unique customers. Most of the previous works concluded that the Ensemble Boosting classifier such as \acs{XGBoost} \& \acs{LGBM} performed better than the traditional machine learning algorithms, with this study, it is reconfirmed on large dataset. On the other hand, this work helped to identify  that the difference in performances between the Deep Learning Network \& Ensemble Boosting Algorithms reduced on large dataset, however the differences were significant in the smaller datasets. Finally, it was reconfirmed with this study that the models performed better on balanced data and  K-Means \acs{SMOTE} is a suitable candidate for fixing the class imbalance problem. 

Additionally, this study proposed a Lean \acs{LGBM} model for the industry use cases. The proposed model provides similar performance as of the \acs{ANN} model but using less data and computation.

\subsection{Limitations \& Future Work}

\subsubsection{Limitations}
Training using some of the traditional machine learning algorithms, such as Non Linear \acs{SVM}, \acf{KNN} etc, were not possible due to the large dataset size and the related time complexity. Feature selection using Sequential Feature Selection technique was tried to select 50 features from the Secondary Dataset;however, due to the large number of features and dataset size, Sequential Feature Selection did not complete even after 20 hours. Thus, due to the time constraint on the project as well as the cost related to training on Google Colab, Sequential Feature Selection was dropped from experiments and instead Select From Model technique was used.\\

Additionally, though multiple data over/under sampling techniques, such as SVMSMOTE, BorderlineSMOTE, Cluster Centroids etc,  were tried, none of them worked as some throw memory overflow issue and others did not complete due to higher time complexity. Since the features of the dataset is anonymized, applying domain knowledge to feature selection process was not possible; moreover, features identified as important in section \ref{subsec:feature_imp} can not be made sense in real world as the features are anonymized. 
Finally, the  data types auto-detected by the Pandas framework while loading dataset is not optimal because even though some of the features contains values between 0 - 1 with precision up to 10 decimal points, still Pandas chooses float64 as the datatype by default. This caused the in-memory size of the dataset to quickly exceed the available memory and making impossible to try some of the machine learning algorithms.

\subsubsection{Future Work}
Scikit Learn \citep{scikit-learn} provides limited support for parallel processing, yet this can be explored to make certain computations faster and it might help to overcome some of the computation related limitations mentioned in previous section. Since the data types auto-detected by the Pandas framework while loading the dataset is not optimal, one can explore individual features to manually find an optimal datatype of the feature and save this new dataset in Parquet format. This will help to reduce the overall size of the dataset and will make loading the dataset to memory much easier.

Attention based Transformer networks recently shown to perform well for time series based classification tasks \citep{wen2022transformers}, \citep{lim2021temporal}, \citep{cholakov2021transformers}, this could be explored on the Primary Dataset to extract dynamic features or could be used as part of the Ensemble Stacking model as one of the learners. TabNet (Attentive Interpretable Tabular Learning)\citep{arik2021tabnet} architecture on classification problems with tabular dataset provides promising performance;thus, this architecture could be explored on the Primary/Secondary dataset to check if it is able  provides better performance than the models explored in this project.

In order to overcome some of the challenges related to feature selection mentioned in the previous section, one can explore the idea of stacking the different feature selection techniques in sequential manner, so that less expensive feature selection technique is applied first and then more expensive techniques applied later in the feature selection pipeline.

Finally, CDGAT \citep{wu2022cdgat} technique discussed in section \ref{sec:literature_review} seems to be promising on credit card default prediction problem, this could be explored on the Primary Dataset to check if the performance can be improved.


\subsection{Summary}
In summary, this section initially discussed the evaluation results of different model on the test set. It was identified that the \acs{ANN} network scored the highest on F1 Score \& Recall metrics while \acs{GRU}+\acs{ANN}+\acs{GBDT} scored the highest score on accuracy metric. The Lean \acs{LGBM} model performed on par with the best scoring models while using less features;also, the Lean \acs{LGBM} scored highest on Precision metric. Top 5 important features were identified from the Lean \acs{LGBM} model and the same was presented in this section.
Then the achievements of the study, such as reconfirming understanding from prior studies on huge industry scale dataset and identifying that the performance of Ensemble Boosting models is on par with the deep learning architecture performance on large datasets etc were discussed.
Finally, the limitations related to the data sampling, feature selection, huge dataset size were discussed along with proposing directions for future work such as exploring TabNet, Transformer Network, CDGAT etc..
\vfill
\clearpage
\section{Conclusion}\label{sec:conclusion}
This study aimed to evaluate the performance of different Machine Learning and Deep Learning techniques on an industry scale dataset for Credit Card Default prediction. The American Express - Default Prediction Dataset \citep{amex-default-prediction-dataset} was chosen as Primary Dataset for the experiments as the dataset contained 5 Million records from year 2018-19; in addition, a Secondary Dataset was created from the Primary Dataset using feature engineering techniques. The dataset was trained on 8 different models(5 Machine Learning \& 3 Deep Learning models); moreover, Early Stopping technique in \acs{ML} \& Dropouts in \acs{DL} were used to avoid overfitting issues of the model. In the pre-processing pipeline of the model, K-Means \acs{SMOTE} Data Oversampling techniques were used to mitigate the Class Imbalance issue. \acs{LGBM} model provided the highest F1-Score of 81.01\% among the \acs{ML} models; on the other hand, \acs{ANN} model scored highest F1-Score of 81.03\% among the \acs{DL} models. Evidently, the performance improvement obtained by using \acs{ANN} is not significant compared to \acs{LGBM} model. Based on this observation, a Lean \acs{LGBM} model was proposed for industry use cases. The proposed model uses K-Means \acs{SMOTE} for Data Oversampling, Select From Model technique for Feature Selection/Dimensionality Reduction, and Gird Search for optimizing the model architecture. The Lean \acs{LGBM} performs on par with the \acs{LGBM} and \acs{ANN} models and provided F1-Score of 80.95\%. 

In conclusion,  the Ensemble Boosting model \acs{LGBM} combined with feature selection and Data Oversampling technique provides the best performance in terms of F1-Score \& computational efficiency for the Credit Card Default Prediction task.



\vfill
\clearpage
\lhead{}\chead{MSc. Project Report :: \nouppercase{\leftmark}}\rhead{}
\phantomsection
\addcontentsline{toc}{section}{References}
\bibliographystyle{agsm} 
\bibliography{mybib}

\vfill
\clearpage
\section{Appendix One: Code}

\subsection{Directory Structure} 

\subsection{Running the Provided Code}

\clearpage
\end{document}


